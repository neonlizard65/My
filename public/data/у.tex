Три бригады слесарей изготовили 1085 деталей. Сколько деталей изготовила каждая бригада отдельно, если известно, что вторая бригада изготовила деталей в 2 раза больше, чем первая, а третья – на 70 деталей меньше, чем вторая?
Катер прошел расстояние между пристанями по течению реки за 2 часа, а обратный путь –за 2,5 часа. Скорость течения реки равна 2 км / ч. Найдите расстояние между пристанями.
Расстояние от А до В 90 км. Пассажир, прибыв поездом в В, пробыл на станции 20 мин, а затем возвратился тем же поездом обратно, затратив в целом 3 ч 40 мин. Найдите скорость электропоезда.
За 6 ч работы ученик сделал столько же деталей, сколько мастер за 4 ч. Известно, что мастер изготовлял в час на 5 деталей больше, чем ученик. Сколько деталей в час изготовлял ученик?


Вариант 2

Решите задачи:

В трех школах 3080 учащихся. В первой школе учащихся в 2 раза меньше, чем во второй, а в третьей на 80 учащихся больше, чем в первой. Сколько учащихся в каждой школе?
Из пункта А со скоростью 60 км / ч выехала грузовая машина. Через 2 ч вслед за грузовой выехала легковая машина со скоростью 80 км / ч. Через сколько часов и на каком расстоянии от А легковая машина догонит грузовую?
Теплоход прошел расстояние между пунктами А и В по течению за 4 ч 30 мин, а из В в А против течения он прошел за 6 ч 18 мин. Какова скорость теплохода в стоячей воде, если скорость течения реки равна 2,4 км / ч?
Два фрезеровщика работали один после другого 32 дня. Первый из них за день производил 25 деталей, а второй – 15 деталей. Сколько дней работал каждый фрезеровщик, если оба они изготовили одинаковое количество деталей?

Вариант 3

Решите задачи:

На ферме было собрано 240 т овощей. В том числе свеклы в три раза меньше, чем моркови, а картошки столько, сколько моркови и свеклы вместе. Сколько собрали картошки, моркови и свеклы?
Расстояние между А и В велосипедист может проехать на 5 ч 20 мин быстрее пешехода. Скорость велосипедиста 12 км / ч , а скорость пешехода 4 км / ч. За какое время велосипедист пройдет путь от А до В?
На соревнованиях по гребле спортсмен 10 мин плыл на лодке вниз по течению реки. На обратный путь против течения он затратил 30 мин. Найдите собственную скорость лодки (в км / ч), если скорость течения реки 2 км / ч.
Мастер изготовляет на 8 деталей в час больше, чем ученик. Ученик работал 6 ч, а мастер
      8 ч, и вместе они изготовили 232 детали. Сколько деталей в час изготовлял ученик?


      1. Стоимость проезда в пригородном электропоезде составляет 198 рублей. Школьникам предоставляется скидка 50%. Сколько рублей стоит проезд группы из 4 взрослых и 12 школьников?


      2. Чашка, которая стоила 90 рублей, продаётся с 10%-й скидкой. При покупке 10 таких чашек покупатель отдал кассиру 1000 рублей. Сколько рублей сдачи он должен получить?
      
      
      3. Альбом, который стоил 120 рублей, продаётся с 25%-ой скидкой. При покупке 5 таких альбомов покупатель отдал кассиру 500 рублей. Сколько рублей сдачи он должен получить?
      
      
      4.. Чайник, который стоил 800 рублей, продаётся с 5%-ой скидкой. При покупке этого чайника покупатель отдал кассиру 1000 рублей. Сколько рублей сдачи он должен получить?
      
      
      5. Набор полотенец, который стоил 200 рублей, продаётся с 3%-й скидкой. При покупке этого набора покупатель отдал кассиру 500 рублей. Сколько рублей сдачи он должен получить?
      
      
      6.Пылесос, который стоил 3500 рублей, продаётся с 10%-й скидкой. При покупке этого пылесоса покупатель отдал кассиру 5000 рублей. Сколько рублей сдачи он должен получить?
      
      
      7. Блюдце, которое стоило 40 рублей, продаётся с 10%-й скидкой. При покупке 10 таких блюдец покупатель отдал кассиру 500 рублей. Сколько рублей сдачи он должен получить?
      
      
      8. Городской бюджет составляет 45 млн. р., а расходы на одну из его статей составили 12,5%. Сколько рублей потрачено на эту статью бюджета?
      
      
      9. Перед представлением в цирк для продажи было заготовлено некоторое количество шариков. Перед началом представления было продано всех воздушных шариков, а в антракте – еще 12 штук. После этого осталась половина всех шариков. Сколько шариков было первоначально?
      
      
      10. Сберегательный банк начисляет на срочный вклад 20% годовых. Вкладчик положил на счет 800 р. Какая сумма будет на этом счете через год, если никаких операций со счетом проводиться не будет?
      
      
      
      
      2-вариант
      
      
      11. Товар на распродаже уценили на 20%, при этом он стал стоить 680 р. Сколько стоил товар до распродажи?
      
      
      12. Государству принадлежит 60% акций предприятия, остальные акции принадлежат частным лицам. Общая прибыль предприятия после уплаты налогов за год составила 40 млн. р. Какая сумма из этой прибыли должна пойти на выплату частным акционерам?
      
      
      13. Акции предприятия распределены между государством и частными лицами в отношении 3:5. Общая прибыль предприятия после уплаты налогов за год составила 32 млн. р. Какая сумма из этой прибыли должна пойти на выплату частным акционерам?
      
      
      
      14. На пост председателя школьного совета претендовали два кандидата. В голосовании приняли участие 120 человек. Голоса между кандидатами распределились в отношении 3:5. Сколько голосов получил победитель?
      
      
      15.Число хвойных деревьев в парке относится к числу лиственных как 1:4. Сколько процентов деревьев в парке составляют лиственные?
      
      
      16.Средний вес мальчиков того же возраста, что и Сергей, равен 48 кг. Вес Сергея составляет 120% среднего веса. Сколько весит Сергей?
      
      
      17. В начале года число абонентов телефонной компании «Север» составляло 200 тыс. чел., а в конце года их стало 210 тыс. чел. На сколько процентов увеличилось за год число абонентов этой компании?
      
      
      18. Тест по математике содержит 30 заданий, из которых 18 заданий по алгебре, остальные  –– по геометрии. В каком отношении содержатся в тесте алгебраические и геометрические задания?
      
       
      
      19. Городской бюджет составляет 45 млн. р., а расходы на одну из его статей составили 12,5%. Сколько рублей потрачено на эту статью бюджета?
      
      
      20. На счет в банке, доход по которому составляет 15% годовых, внесли 24 тыс. р. Сколько тысяч рублей будет на этом счете через год, если никаких операций со счетом проводиться не будет?
      
      
      
      
      
      3-вариант
      
      
      21. Какая сумма (в рублях) будет проставлена в кассовом чеке, если стоимость товара 520 р., и покупатель оплачивает его по дисконтной карте с 5%-ной скидкой?
      
      
      22.Товар на распродаже уценили на 20%, при этом он стал стоить 680 р. Сколько рублей стоил товар до распродажи?
      
      
      23.. На пост председателя школьного совета претендовали два кандидата. В голосовании приняли участие 120 человек. Голоса между кандидатами распределились в отношении 3:5. Сколько голосов получил победитель?
      
      
      24. Число хвойных деревьев в парке относится к числу лиственных как 1:4. Сколько процентов деревьев в парке составляют лиственные?
      
      
      25. В понедельник некоторый товар поступил в продажу по цене 1000 р. В соответствии с принятыми в магазине правилами цена товара в течение недели остается неизменной, а в первый день каждой следующей недели снижается на 20% от предыдущей цены. Сколько рублей будет стоить товар на двенадцатый день после поступления в продажу?
      
      
      26. Средний вес мальчиков того же возраста, что и Сергей, равен 48 кг. Вес Сергея составляет 120% среднего веса. Сколько килограммов весит Сергей?
      
      27. Стоимость проезда в пригородном электропоезде составляет 198 рублей. Школьникам предоставляется скидка 50%. Сколько рублей стоит проезд группы из 4 взрослых и 12 школьников?
      
      
      28. Поступивший в продажу в январе мобильный телефон стоил 3000 рублей. В марте он стал стоить 2790 рублей. На сколько процентов снизилась цена на мобильный телефон в период с января по март?
      
      
      29. Поступивший в продажу в январе мобильный телефон стоил 3000 рублей. В апреле он стал стоить 2160 рублей. На сколько процентов снизилась цена на мобильный телефон в период с января по апрель?
      
      
      30. Клубника стоит 180 рублей за килограмм, а клюква — 250 рублей за килограмм. На сколько процентов клубника дешевле клюквы?
      
      
      
      
      
      
      
      4-вариант
      
      
      31. Виноград стоит 160 рублей за килограмм, а малина — 200 рублей за килограмм. На сколько процентов виноград дешевле малины?
      
      
      32. На многопредметной олимпиаде всех участников получили дипломы,  остальных участников были награждены похвальными грамотами, а остальные 144 человека получили сертификаты об участии. Сколько человек участвовало в олимпиаде?
      
      
      33.На складе есть коробки с ручками двух цветов: чёрные и синие. Коробок с чёрными ручками 4, с синими — 11. Сколько всего ручек на складе, если чёрных ручек 640, коробки одинаковые и в каждой коробке находятся ручки только одного цвета?
      
      
      34. Кисть, которая стоила 240 рублей, продаётся с 25%-й скидкой. При покупке двух таких кистей покупатель отдал кассиру 500 рублей. Сколько рублей сдачи он должен получить?
      
      
      35. Площадь земель крестьянского хозяйства, отведённая под посадку сельскохозяйственных культур, составляет 24 га и распределена между зерновыми и овощными культурами в отношении 5:3. Сколько гектаров занимают овощные культуры?
      
      
      36. Спортивный магазин проводит акцию: «Любая футболка по цене 300 рублей. При покупке двух футболок — скидка на вторую 60%». Сколько рублей придётся заплатить за покупку двух футболок?
      
      
      37. В течение августа помидоры подешевели на 50%, а затем в течение сентября подорожали на 70%. Какая цена меньше: в начале августа или в конце сентября — и на сколько процентов?
      
      
      38. Клубника стоит 180 рублей за килограмм, а клюква — 250 рублей за килограмм. На сколько процентов клубника дешевле клюквы?
      
      
      39. Виноград стоит 160 рублей за килограмм, а малина — 200 рублей за килограмм. На сколько процентов виноград дешевле малины?
      
      
      40. Поступивший в продажу в апреле мобильный телефон стоил 4000 рублей. В сентябре он стал стоить 2560 рублей. На сколько процентов снизилась цена на мобильный телефон в период с апреля по сентябрь?
      
      
    